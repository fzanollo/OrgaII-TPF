\section{Implementación}
El método requiere de los siguientes datos de entrada:
\begin{description}
\item[divisions] cantidad de divisiones sobre la cuál se colocarán los picos.
\item[nroPeaks] cantidad total de picos.
\item[yMin] altura mínima permitida de un pico.
\item[yMin] altura máxima permitida de un pico.
\end{description}
\smallskip

En nuestro caso también contamos con las siguientes entradas, las cuáles agregamos por razones de utilidad a la hora de experimentar y comparar las implementaciones de C y ASM.
\begin{description}
\item[seed] permite setear una semilla particular para el random, esto es para lograr el mismo gráfico y poder experimentar con datos más certeros.
\item[debugging] si una semilla es proporcionada entonces se le puede decir al programa que entre en modo verbouse; el cuál nos va a dar, además del gráfico, el valor numérico de cada posición del terrreno final y la posición y tamaño original de cada pico.
\end{description}
\smallskip

Para poder llevar a cabo el método vamos a estar utilizando las siguientes estructuras:
\begin{description}
\item[peaksPos] arreglo con nroPeaks posiciones, cada una contiene la posición dentro del terreno de dicho pico. Las posiciones van de 0 a divisions-1.
\item[peaksSize] arreglo con nroPeaks posiciones, cada una contiene la altura de dicho pico. Las alturas posible están entre yMin e yMax (inclusive en ambos casos).
\item[terrain] vector con los valores numéricos finales de cada posición del terreno.
\end{description}
\smallskip

Los picos son generados de manera aleatoria, tanto su posición como su altura (dentro de los límites explicados más arriba). Luego se hace lo siguiente:

\begin{algorithm}
\begin{algorithmic}

\ForAll{positions} 
	\ForAll{peaks} 
		\Comment{se calcula la influencia de cada pico para cada posición}
		\State influencia $\gets$ altura del pico - distancia del pico a la posición actual * ruggness.
	\EndFor
	
	\If{la posición no tiene influencia de nadie} 
		\State valor final de la posición $\gets$ 0.
	
	\ElsIf{la posición solo es influenciada por un pico} 
		\State valor final de la posición $\gets$ influencia / 2.
	
	\ElsIf{la posición es influenciada por dos o más picos}
		\State valor final de la posición $\gets$ influencia / cantidad de picos con influencia sobre ella.
		
	\EndIf

\EndFor

\end{algorithmic}
\end{algorithm}
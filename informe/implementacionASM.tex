\subsection{Versión de ASM}
La idea es simple, recorrer las posiciones del terreno de a cuatro, ya que como cada una es un float esa es la cantidad que entran en un registro xmm. 
Luego, en cada iteración, recorrer todos los picos, también de a cuatro (son int pero de 32 bits, es decir, cuatro por registro xmm).

Cuando obtenemos los datos de los picos, expandimos y repetimos la información de cada uno en distintos registros. 
Para que la explicación sea más clara mostraremos el caso del pico uno (P1) siendo los demás exactamente iguales salvo por la información a usar. Entonces:

Primero, tanto para la posición como para la altura de P1, se crean registros con la información replicada en cada int.

GRAFICO CON ESTO:
|P1|P2|P3|P4|
flechitas
|P1|P1|P1|P1|

Luego

